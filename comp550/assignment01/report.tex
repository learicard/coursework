%&pdflatex
%% filename: amsart-template.tex, version: 2.1
\documentclass{amsart}
\usepackage{hyperref}
\usepackage{inputenc}
\usepackage{graphicx}
\usepackage{bbm}
\usepackage{csvsimple}

\newtheorem{theorem}{Theorem}[section]
\newtheorem{lemma}[theorem]{Lemma}
\theoremstyle{definition}
\newtheorem{definition}[theorem]{Definition}
\newtheorem{example}[theorem]{Example}
\newtheorem{xca}[theorem]{Exercise}
\theoremstyle{remark}
\newtheorem{remark}[theorem]{Remark}
\numberwithin{equation}{section}
\setlength{\parindent}{0pt} % turn off auto-indent

\graphicspath{ {./} }

\begin{document}

\title{Assignment 1: [COMP550]}

\author{Joseph D. Viviano}
\address{McGill University}
\curraddr{}
\email{joseph@viviano.ca}
\thanks{}
\date{Sept 2018}

\maketitle

\section{Ambiguity}

% Phonetics
% Discourse
\subsection{Phonology}

\url{https://books.google.ca/books?id=eJUvDGXRW7AC&pg=PA92#v=onepage&q&f=false} \\

"Mr Buyer, I understand that you do not have the money to write me a check for the retainer tonight. When would it be all \textbf{right} for you to \textbf{write} me a check for the retainer?" \\

\textit{Right} and \textit{write} sound identical in English, but have very different meanings. Indeed, \textit{right} after the word "all" means something potentially different than \textit{right} on it's own. Here, \textit{right} means "correct", whereas \textit{write} mean "to use a hand-held utensil for the purposes of placing words on a medium such as paper". For a reader to disambiguate these words, one could look to the word before right, i.e., "all right", to see that right in this case means correct, where as one could look at the words directly after "write", i.e., "write me a check", to see that this word refers to the act of visual letter production. \\

\subsection{Morphology}

\url{https://twitter.com/JVM/status/1020084244490670080} \\

"We are \textbf{live} in New York City’s Brooklyn Bridge Park at the high-profile photography exhibit The Fence 2018, an exciting photo series with an animal rights message starring rescued chickens." \\

\textit{Live} by itself 'happening right now' or 'a place that I call home', and are written the same way although they sound differently when spoken. The reader must use the words immediately before it, "we are" to determine that this word "live" refers to "happening right now". \\


\subsection{Syntax}

\url{http://journals.sagepub.com/doi/abs/10.1177/1744987109358836?journalCode=jrnb} \\

"The aims of the present study were both to explore how the oldest of \textbf{old} men and women with estimated high resilience talk about experiences of becoming and being old, and to discuss the analysis of their narratives in terms of the foundational concepts of the Resilience Scale (RS)." \\

\textit{Old} in this case implies that the men are old, but it isn't clear whether this sentence applies only to old women, or all women with an estimated high resiliance. A non-ambiguous version of this sentence would be "oldest of old men and old women". In this case, the reader would have to look further to "talk about experiences of becoming and being old" to infer that all of the women being discussed here are old. \\

\subsection{Semantics}

\url{https://en.wikipedia.org/wiki/Domestication_of_the_Syrian_hamster#Capture_of_live_hamsters} \\

"The domestication of \textbf{the Syrian hamster} began in the late 1700s when naturalists cataloged the Syrian hamster, also known as Mesocricetus auratus or the golden hamster." \\

\textit{The} in this case could refer to either a single hamster of the Syrian species, or the entire species of Syrian hamsters. A non-ambiguous version of the sentence would explicity use the word 'species' i.e., "The domestication of the Syrian hamster species began in the late 1700s". To disambiguate these meanings, the reader would have to have knowledge of the most probable use of "the Syrian hamster" in English, or knowledge from the sentences before (e.g., a particular hamster was introduced). \\

\subsection{Pragmatics}

\url{https://grassrootsmotorsports.com/forum/off-topic-discussion/i-hate-squirrels/33189/page2/} \\

"\textbf{I chased} a squirrel around the yard. Up and down the yard, back and forth, back and forth and the squirrel ran up a tree. So now the car's totaled. -- Emo Phillips (a joke)." \\

\textit{I chased} without context suggests a person persuing a squirrel around the yard on foot, because that is most likely, but in fact any method of pursuit could have been employed because no particular method was defined. In this case, the reader must continue to the punchline, which makes the meaning of the first sentence concrete, i.e., the person was in an automobile. This is a cause for levity in some. \\


\section{FST for Spanish Verbal Conjugation}
    \begin{table}[ht]
    \centering
    \resizebox{\textwidth}{!}{\begin{tabular}{l l|l|l|l|l|l}%
    \bfseries Infinitive & \bfseries 1st sg & \bfseries 2nd sg & \bfseries 3rd sg & \bfseries 1st pl & \bfseries 2nd pl & \bfseries 3rd pl % specify table head
    \csvreader[head to column names]{lex_table.csv}{}% use head of csv as column names
    {\\\hline\csvcoli&\csvcolii&\csvcoliii&\csvcoliv&\csvcolv&\csvcolvi&\csvcolvii}% specify your coloumns here
    \end{tabular}}
\end{table}
%infinitive,1-sg-reg,2-sg-reg,3-sg-reg,1-pl-reg,2-pl-reg,3-pl-reg


\end{document}
