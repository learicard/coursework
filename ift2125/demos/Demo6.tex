\documentclass[11pt]{article} % use larger type; default would be 10pt
\usepackage{geometry} % to change the page dimensions
\geometry{a4paper} % or letterpaper (US) or a5paper or....
%%% PACKAGES
\usepackage{booktabs} % for much better looking tables
\usepackage{array} % for better arrays (eg matrices) in maths
\usepackage{paralist} % very flexible & customisable lists (eg. enumerate/itemize, etc.)
\usepackage{verbatim} 
\usepackage{amsmath}
\usepackage{amssymb}
\usepackage{graphicx}
\usepackage[utf8]{inputenc}
\setlength{\parindent}{0ex}
\setlength{\parskip}{1em}

\usepackage{tikz}
\usetikzlibrary{arrows,shapes}

\usepackage{listings}
\usepackage{color}

\definecolor{dkgreen}{rgb}{0,0.6,0}
\definecolor{gray}{rgb}{0.5,0.5,0.5}
\definecolor{mauve}{rgb}{0.58,0,0.82}

\lstset{frame=tb,
	language=Python,
	aboveskip=3mm,
	belowskip=3mm,
	showstringspaces=false,
	columns=flexible,
	basicstyle={\small\ttfamily},
	numbers=none,
	numberstyle=\tiny\color{gray},
	keywordstyle=\color{blue},
	commentstyle=\color{dkgreen},
	stringstyle=\color{mauve},
	breaklines=true,
	breakatwhitespace=true,
	tabsize=3
}

\newenvironment{question}[1][\unskip]{%
	\par
	\noindent
	\textbf{Question #1:}
	\noindent}
{\medskip}

\newenvironment{solution}[1][\unskip]{%
	\par
	\noindent
	\textbf{Solution #1:}
	\noindent}
{\medskip}

\begin{document}
		
	\leftline{\bfseries Introduction à l'algorithmique \hfill Hiver 2018}
	
	\noindent \hrulefill
	\leftline{IFT2125-6001 \hfill TA: Stéphanie Larocque}
	\centerline{\bfseries Démonstration 6}
	\centerline{À partir des corrigés de Maelle Zimmermann}	
	\noindent \hrulefill
	
	\vspace{1cm}
	
	\section{}
	\begin{question}
		Exécuter l'algorithme de Prim sur le graphe suivant:
		\begin{center}
			\includegraphics[]{Prim.png}
		\end{center}
	\end{question}
	\begin{solution}
		En exécutant l'algorithme, nous obtenons:
		\begin{center}
			\begin{tabular}{c|c|c}
				Itération & $(u,v)$ & $B$ \\
				\hline
				0 & - & $\{1\}$\\
				1 & $(1,2)$ & $\{1,2\}$\\
				2 & $(2,3)$ & $\{1,2,3\}$ \\
				3 & $(1,4)$ & $\{1,2,3,4\}$ \\
				4 & $(4,5)$ & $\{1,2,3,4,5\}$ \\
				5 & $(4,7)$ & $\{1,2,3,4,5,7\}$ \\
				6 & $(7,6)$ & $\{1,2,3,4,5,7,6\}$
			\end{tabular}
		\end{center}
		Ainsi l'arbre sous-tendant de poids minimal est de poids 17.
	\end{solution}
	
	\section{}
	\begin{question}
		Montrer que l'algorithme de Prim peut, comme celui de Kruskal, être implémenté en utilisant des monceaux. Montrer qu'il prend alors un temps dans $\Theta(a \log n)$.
	\end{question}
	
	\begin{solution}
		Considérons d'abord l'algorithme de Prim implémenté naïvement sans monceaux:
		\begin{lstlisting}
		def Prim(V, E):
			def poids((u, v, c)): return c
			F = sorted(E, key = poids)
			T = []
			B = set(V[:1])
		
			# tant que tous les sommets ne sont pas couverts
			while len(B) != len(V)
				for (u, v, _) in F:
					if (u in B) != (v in B):
						break
				T.append((u, v))
				B.update([u, v])
			return T
		\end{lstlisting}
		Dans le pire cas, cet algorithme prend un temps dans $O(an)$ (boucle while exécutée exactement $n-1$ fois et parcours de $F$ en entier qui contient $a$ arêtes). Mais il existe une meilleure implémentation.
		Voici l'algorithme de Prim implémenté avec des monceaux:
		\begin{lstlisting}
		def Prim_heap(V, E):
			if len(V) == 0:
				return []
			x = V[0] # sommet actuel
			T = [] # arbre partiel minimum
			B = set(V[:1]) # sommets couverts par T
			H = [] # monceau vide
		
			# construire voisins des sommets
			voisins = [[] for v in V]
			for (u, v, c) in E:
				voisins[u].append((v, c))
				voisins[v].append((u, c))
		
			# calcul de l'arbre
			while len(T) < len(V) - 1:
			# met tous les voisins de x dans le monceau
				for (y, c) in voisins[x]:
					heappush(H, (c, (x, y)))
		
				# retire l'arete au poids c minimum
				(c, (u, v)) = heappop(H)
		
				# continue de retirer jusqu'a ce que l'arete traverse B et V\B
				while (u in B) == (v in B):
					(c, (u, v)) = heappop(H)
				
				# update x, T et B
				x = u if u not in B else v
				T.append((u, v))
				B.add(x)
			return T
		\end{lstlisting}
		Dans un monceau, les opérations push et pop prennent un temps dans $\Theta(\log k)$ et $\Theta(\log k)$ respectivement (où $k$ est le nombre d'éléments dans le monceau). 
		Posons $n=|V|$ et $a=|E|$, alors:
		\begin{itemize}
			\item {La boucle qui construit les voisins est exécutée $a$ fois donc prend un temps dans $\Theta(a)$.}
			\item {Chaque arête $(u,v)$ est ajoutée au plus 2 fois dans le monceau, soit via $u$ soit via $v$. Il y a donc au plus $2a$ opérations push qui nécessitent chacune un temps de $\Theta(\log a)$. Donc le temps total des ajouts est dans $\Theta(a \log a)$.}
			\item{Similairement, il y a au plus $2a$ opérations pop qui nécessitent chacune un temps de $\Theta(\log a)$. Donc le temps total des retraits est dans $\Theta(a \log a)$.} 
			\item{Les opérations append et add sont exécutées autant de fois qu'il y a d'itérations de l'algorithme, donc au plus $n-1$ fois.}
		\end{itemize}
		Le temps d'exécution de l'algorithme est donc dans $\Theta(a \log a)$, ce qui s'écrit aussi $\Theta(a \log n)$ pour les mêmes raisons que lors de l'analyse de l'algorithme de Kruskal.
	\end{solution}
\end{document}